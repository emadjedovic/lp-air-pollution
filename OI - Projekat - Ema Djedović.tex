\documentclass[12pt, a4paper]{paper}
\usepackage[utf8]{inputenc}
\usepackage[croatian]{babel}
\usepackage{amsmath}
\usepackage{amssymb}
\usepackage[hidelinks]{hyperref}
\usepackage{url}
\usepackage{fancyhdr}
\usepackage{setspace}
\setstretch{1.15}
\setlength{\parindent}{0pt}

\title
{\Huge Projekat iz predmeta\\
Operaciona istraživanja\bigskip}
\subtitle{\Large Tema: Linearni program za minimizaciju zagađenja zraka uz optimizaciju troškova s osvrtom na društveni aspekt}
\author{Ema Djedović \\
Broj indeksa: 6009/M \\
Smjer: Kompjuterske nauke \\
PMF Sarajevo\\
12/2023}

\pagestyle{fancy}
\fancyhf{}
\lhead{Ema Djedović}
\chead{}
\rhead{\thepage}

\begin{document}
\maketitle
\thispagestyle{empty}

\newpage
\tableofcontents
\thispagestyle{empty}

\newpage
\clearpage
\pagenumbering{arabic}

\section*{UVOD}
\addcontentsline{toc}{section}{Uvod}

\paragraph{}
U današnjem svijetu, energetski sektor igra ključnu ulogu u ostvarivanju ravnoteže između zadovoljenja energetskih potreba i smanjenja negativnih uticaja na životnu sredinu. CO$_2$ je jedan od glavnih plinova s efektom staklenika koji zadržava toplotu u atmosferi, doprinoseći povećanju temperature na Zemlji. Ova povećana temperatura izaziva niz nepoželjnih posljedica:

\begin{itemize}
	\item Rast nivoa mora: topljenje ledenih ploča, ugrožene obale i niske oblasti.
	\item Ekstremni vremenski događaji: suše, poplave, oluje i toplotni valovi.
	\item Gubitak biodiverziteta: ugrožene mnoge vrste.
	\item Promjene u poljoprivredi: smanjenje prinosa i ugrožavanje globalne sigurnosti hrane.
\end{itemize}

\paragraph{}
Smanjenjem emisija CO$_2$ doprinosimo globalnim naporima da se uspore ove štetne posljedice. Ovaj projekat ima za cilj razvoj linearnog programa koji se fokusira na minimizaciju zagađenja zraka u Bosni i Hercegovini uz istovremenu optimizaciju troškova, a s osvrtom na društveni aspekt.

\paragraph{}
Koristeći alate operacionih istraživanja, analiziramo optimalne godišnje količine različitih energenata, uzimajući u obzir emisione faktore i ograničenja dostupnih resursa. U drugom koraku uvođenjem alternativnih izvora energije istražujemo mogućnosti postizanja održivijeg energetskog bilansa.
\paragraph{}
U trećem koraku, uvodimo cijene klasičnih i alternativnih izvora energije, minimizirajući ukupne troškove uzimajući u obzir instalaciju i održavanje elemenata alternativnih izvora.
\paragraph{}
Kroz četvrti korak, projekt dodatno uzima u obzir društveni aspekt, maksimizirajući broj zaposlenih na godišnjem nivou na različitim proizvodnim i instalacionim pogonima.

\newpage
\section*{KORAK I}
\addcontentsline{toc}{section}{Korak I}

\subsubsection*{a) Varijable odluke}

Želimo pronaći optimalne godišnje količine energenata:\\

$x_1$ - količina uglja (u kilogramima)\\
$x_2$ - količina prirodnog plina (u kubnim metrima)\\
$x_3$ - količina ogrjevnog drveta (u kilogramima)\\
$x_4$ - količina drvnog peleta (u tonama)

\subsubsection*{b) Funkcija cilja}

Prema podacima Međunarodne agencije za energiju (IEA), možemo odrediti emisione faktore u vidu koeficijenata $p_i$. Emisioni faktor je kvantitivna mjera koja izražava količinu otpuštenog zagađivača u atmosferu. Za ugalj, taj faktor varira u rasponu od $1.7$ do $3.6$ kilograma CO$_2$ po kilogramu sagorenog uglja. Prema IEA, prosječni emisioni faktor uglja za Bosnu i Hercegovinu iznosi $2.4$. Sa $p_1, p_2, p_3$ i $p_4$ označimo respektivno emisione faktore za ugalj, prirodni plin, ogrjevno drvo i drvni pelet. Imamo:
\begin{align*}
p_1 &= 2.4 \quad \text{(ugalj)} \\
p_2 &= 1.91 \quad \text{(prirodni plin)} \\
p_3 &= 1.72 \quad \text{(ogrjevno drvo)} \\
p_4 &= 1.5 \quad \text{(drvni pelet)}
\end{align*}

Nivo zagađenja zraka je izražen u kilogramima otpuštenog CO$_2$, a predstavljamo ga preko jednadžbe:
\[ \text{zagađenje} = p_1 \cdot x_1 + p_2 \cdot x_2 + p_3 \cdot x_3 + p_4 \cdot x_4 \]

Naša funkcija cilja je:
\[ min(\text{zagađenje}) = p_1 \cdot x_1 + p_2 \cdot x_2 + p_3 \cdot x_3 + p_4 \cdot x_4 \]

\subsubsection*{c) Ograničenja}

količina dostupnog uglja: $x_1 \leq 225,000,000$ kg \\
količina dostupnog prirodnog plina: $x_2 \leq 254,790,000$ m$^3$ \\
količina dostupnog ogrjevnog drveta: $x_3 \leq 245,850$ kg \\
količina dostupnog drvnog peleta: $x_4 \leq 300,000,000$ kg \\

A, B, C, D - količina proizvedene topline
\\\\
$1$ kg uglja = $8$ kWh toplotne energije (A = $8$ kWh/kg) \\
$1$ m$^3$ prirodnog plina = $3.5$ kWh toplotne energije (B = $3.5$ kWh/m$^3$) \\
$1$ kg ogrjevnog drveta = $3$ kWh toplotne energije (C = $3$ kWh/kg) \\
$1$ kg drvnog peleta = $4.85$ kWh toplotne energije (D = $4.85$ kWh/kg) \\

$A \cdot x_1 + B \cdot x_2 + C \cdot x_3 + D \cdot x_4 \geq H$ \\
$H$ - ukupna godišnja potreba za toplotnom energijom $6.000.000$ kWh\\

$x_1, x_2, x_3, x_4 \geq 0$ (nenegativnost) \\

Neke zemlje, regije i organizacije imaju regulative koje se odnose na specifične industrije i sektore u cilju rada na ublažavanju efekta globalnog zatopljenja, ali ne postoji zakonska gornja granica za emisiju zagađivača izraženu u kilogramima CO$_2$.

\newpage
\section*{KORAK II}
\addcontentsline{toc}{section}{Korak II}

Sada uvodimo pogone alternativnih izvora energije. Električnu energiju koju koristimo za zagrijavanje objekata a koju dobijemo u tim pogonima smatramo "čistom", odnosno smatramo da nemaju faktor zagađenja ($p_i = 0$).

\subsubsection*{a) Varijable odluke}

Varijable odluke za alternativne izvore bi bile optimalne količine objekata koje treba instalirati, odnosno broj vjetrenjača, broj solarnih panela i broj hidroelektrana.
\begin{align*}
    x_5 &= \text{broj vjetrenjača za instalirati} \quad (p_5 = 0) \\
    x_6 &= \text{broj solarnih panela za instalirati} \quad (p_6 = 0) \\
    x_7 &= \text{broj hidroelektrana za instalirati} \quad (p_7 = 0)
\end{align*}

\subsubsection*{b) Funkcija cilja}

Kako pri minimizaciji funkcije nivoa zagađenja uzimamo da alternativni izvori energije imaju emisioni faktor $p_i$ nula (ne zagađuju), tako funkcija cilja ostaje ista kao u koraku I:
\[ min(\text{zagađenje}) = p_1 \cdot x_1 + p_2 \cdot x_2 + p_3 \cdot x_3 + p_4 \cdot x_4 \]

\subsubsection*{c) Ograničenja}

Od koristi je istražiti koja su to bosansko-hercegovačka područja koja imaju potencijal za postavljanje ovih objekata. Nakon toga, stvaraju se ograničenja koja bi se ticala količine objekata koje je moguće instalirati na jednu lokaciju, a ovise od raspoloživog prostora, zakonske regulative itd. To su jednostavna ograničenja oblika $x_i \leq n$, $i=5,6,7$, $n \in \mathbb{N}$. Za svrhe modeliranja nama je dovoljno da ostavimo taj opći oblik nejednakosti.\\

Nadalje, konstante E, F i G će sada predstavljati količinu proizvedene energije vjetrenjača, solarnih panela i hidroelektrana (respektivno). Te konstante u ovom trenutku ne možemo vjerodostojno odrediti zbog nepredvidljivosti koje su neminovne u radu s alternativnim izvorima. To su, naprimjer, brzina vjetra, potencijal vode i količina sunčeve svjetlosti. Količina energije koja će se proizvesti u tim pogonima također ovisi od snage i efikasnosti postrojenja (Najveća vjetrenjača na svijetu ima snagu 16 MW, dok je standard u prosjeku 2-3 MW po vjetrenjači.)\\

Ukupna godišnja potreba za toplotnom energijom ($H$) ostaje ista.
\[ A \cdot x_1 + B \cdot x_2 + C \cdot x_3 + D \cdot x_4 + E \cdot x_5 + F \cdot x_6 + G \cdot x_7 \geq H \]
\[ x_1, x_2, x_3, x_4, x_5, x_6, x_7 \geq 0 \text{ (nenegativnost)}\]


\newpage
\section*{KORAK III}
\addcontentsline{toc}{section}{Korak III}

Sada ćemo uzeti u obzir cijene klasičnih energenata te cijene alternativnih izvora na način da minimiziramo troškove za idućih pet godina. Za alternativne izvore (vjetrenjače, solarne panele i hidroelektrane) razmatramo:
\begin{enumerate}
    \item Cijene instalacije elemenata (uzimaju se u obzir samo prve godine)
    \item Cijene održavanja elemenata (konstantne za svaku godinu)
\end{enumerate}

Posmatramo cijenu po jedinici mjere:
\begin{align*}
    c_1 & = \text{cijena uglja (po kilogramu)} \\
    c_2 & = \text{cijena prirodnog plina (po metru kubnom)} \\
    c_3 & = \text{cijena ogrjevnog drveta (po kilogramu)} \\
    c_4 & = \text{cijena drvnog peleta (po toni)} \\
    c_{51} & = \text{cijena instalacije jedne vjetrenjače} \\
    c_{61} & = \text{cijena instalacije jednog solarnog panela} \\
    c_{71} & = \text{cijena instalacije jedne hidroelektrane} \\
    c_5 & = \text{cijena održavanja jedne vjetrenjače za jednu godinu} \\
    c_6 & = \text{cijena održavanja jednog solarnog panela za jednu godinu} \\
    c_7 & = \text{cijena održavanja jedne hidroelektrane za jednu godinu}
\end{align*}

\subsubsection*{a) Varijable odluke}

Ostaju iste kao u koraku II ($x_1\cdots x_7$).

\subsubsection*{b) Funkcija cilja}

Sada je funkcija koju minimiziramo dvokomponentna. Uz minimizaciju funkcije \textit{zagađenje} želimo minimizirati i funkciju \textit{troškovi}:
\[min F = w_1 \cdot \text{zagađenje} + w_2 \cdot \text{troškovi}\]
\[\text{zagađenje} = p_1 \cdot x_1 + p_2 \cdot x_2 + p_3 \cdot x_3 + p_4 \cdot x_4\]
\[
\begin{split}
    \text{troškovi} & = 5 \cdot (c_1 \cdot x_1 + c_2 \cdot x_2 + c_3 \cdot x_3 + c_4 \cdot x_4 \\
    & \quad + c_5 \cdot x_5 + c_6 \cdot x_6 + c_7 \cdot x_7) + c_{51} \cdot x_5 + c_{61} \cdot x_6 + c_{71} \cdot x_7
\end{split}
\]

Faktori težine $w_1$ i $w_2$ odražavaju važnost svakog cilja u problemu. Na primjer, možemo uzeti da je smanjenje zagađenja važnije od smanjenja troškova, pa će faktor težine $w_1$ biti veći od faktora težine $w_2$. Ovo zavisi od naših prioriteta.

\subsubsection*{c) Ograničenja}

Sasvim je prirodno uvesti ograničenja koja se tiču planiranog budžeta za određeni energetski sektor. To su uslovi koji se nameću od strane nadležnih, a podrazumijevaju količinu novca koji su spremni uložiti u koji pogon. Recimo, količina sagrađenih hidroelektrana $(x_7)$ će dakako ovisiti od predviđenog budžeta za iste. Slično posmatramo sve komponente modela.
\begin{align*}
	c_1 \cdot x_1 &\leq \text{budzet\_ugalj}\\
	c_2 \cdot x_2 &\leq \text{budzet\_plin}\\
	c_3 \cdot x_3 &\leq \text{budzet\_drvo}\\
	c_4 \cdot x_4 &\leq \text{budzet\_pelet}\\
	(c_{51}+c_5) \cdot x_5 &\leq \text{budzet\_vjetrenjace}\\
	(c_{61}+c_6) \cdot x_6 &\leq \text{budzet\_panele}\\
	(c_{71}+c_7) \cdot x_7 &\leq \text{budzet\_hidroelektrane}
\end{align*}

\newpage
\section*{KORAK IV}
\addcontentsline{toc}{section}{Korak IV}

Naposljetku, dotičemo se društvenog aspekta te uvodimo varijable broja zaposlenih radnika na jednom pogonu.

\subsubsection*{a) Varijable odluke}

Neka su $x_{8}$, $x_{9}$, $x_{10}$, $x_{11}$, $x_{12}$, $x_{13}$ i $x_{14}$ varijable koje predstavljaju broj zaposlenih radnika na pogonima za proizvodnju, respektivno za ugalj, prirodni plin, ogrjevno drvo, drvni pelet, vjetrenjače, solarne panele i hidroelektrane.\\

Na prvi pogled nema potrebe za korištenjem sedam različitih varijabli ukoliko ravnopravno tretiramo radno mjesto na pogonu za ugalj i onog na nekom drugom pogonu. Međutim, ukoliko svaku varijablu posmatramo individualno, onda nam ostaje sloboda da naknadno uvedemo koeficijente koji će predstavljati plate na svakoj od pozicija, a koje se međusobno razlikuju. U kombinaciji s korakom III, mogla bi se postaviti nova ograničenja vezana za budžet, nadograditi komponentu "troškovi" u funkciji cilja koja će imati za cilj da minimizira plate radnika, postaviti ograničenja minimalne plate itd.

\subsubsection*{b) Funkcija cilja}

Za sada, želimo samo maksimizirati broj radnih mjesta, odnosno \[max(\text{zaposleni}) = x_{8} + x_{9} + x_{10} + x_{11} + x_{12} + x_{13} + x_{14}\]
Ažurirana funkcija cilja s težinskim faktorima $w_1$, $w_2$ i $w_3$ izgleda ovako:
\[min F = w_1 \cdot \text{zagađenje} + w_2 \cdot \text{troškovi} - w_3 \cdot \text{zaposleni}\]

\subsubsection*{c) Ograničenja}

S društvenog aspekta želimo postaviti ograničenja koja se tiču minimalnog broja zaposlenih. Osim toga, imamo ograničenja koja nam govore koliko je radne snage potrebno na kojem pogonu.
\[ x_{8} + x_{9} + x_{10} + x_{11} + x_{12} + x_{13} + x_{14} \geq \text{minimalni\_broj\_zaposlenih} \]

Imamo informacije da održavanje 20 vjetrenjača u prosjeku podrazumijeva dva do tri radna mjesta, za održavanje 60 solarnih panela jedno ili dva radna mjesta, dok jedna hidroelektrana zapošljava između 75 i 196 radnika. Možemo formirati odgovarajuće jednadžbe:
\begin{align*}
	2 \cdot \frac{x_5}{20} \leq\enspace&x_{12}\space\leq 3 \cdot \frac{x_5}{20}\\
	\frac{x_6}{60} \leq\enspace&x_{13}\space\leq 2\cdot \frac{x_6}{60}\\
	75 \cdot x_7 \leq\enspace&x_{14}\space\leq 196 \cdot x_7
\end{align*}

$x_5$ - broj vjetrenjača\\
$x_6$ - broj solarnih panela\\
$x_7$ - broj hidroelektrana\\
$x_{12}$ - broj zaposlenih po jednoj vjetrenjači\\
$x_{13}$ - broj zaposlenih po jednoj solarnoj paneli\\
$x_{14}$ - broj zaposlenih po jednoj hidroelektrani
\[ x_{8}, x_{9}, x_{10}, x_{11}, x_{12}, x_{13}, x_{14} \geq 0\quad \text{(nenegativnost)}\]

\newpage
\section*{ZAKLJUČAK}
\addcontentsline{toc}{section}{Zaključak}

Konačno možemo predstaviti linearni program. Sumiramo sve do sad rečeno i predstavljamo varijable odluke, funkciju cilja i ograničenja.
\subsection*{Varijable odluke}
\begin{align*}
	x_1 &: \text{količina uglja (kg)} \\
	x_2 &: \text{količina prirodnog plina (m}^3\text{)} \\
	x_3 &: \text{količina ogrjevnog drveta (kg)} \\
	x_4 &: \text{količina drvnog peleta (t)} \\
	x_5 &: \text{broj vjetrenjača za instalirati} \\
	x_6 &: \text{broj solarnih panela za instalirati} \\
	x_7 &: \text{broj hidroelektrana za instalirati} \\
	x_8 &: \text{broj zaposlenih na pogonu za proizvodnju uglja} \\
	x_9 &: \text{broj zaposlenih na pogonu za proizvodnju prirodnog plina} \\
	x_{10} &: \text{broj zaposlenih na pogonu za proizvodnju ogrjevnog drveta} \\
	x_{11} &: \text{broj zaposlenih na pogonu za proizvodnju drvnog peleta} \\
	x_{12} &: \text{broj zaposlenih po jednoj vjetrenjači} \\
	x_{13} &: \text{broj zaposlenih po jednom solarnom panelu} \\
	x_{14} &: \text{broj zaposlenih po jednoj hidroelektrani}
\end{align*}

\subsection*{Funkcija cilja}
\begin{align*}
	min F &= w_1 \cdot \text{zagađenje} + w_2 \cdot \text{troškovi} - w_3 \cdot \text{zaposleni} \\
	\text{zagađenje} &= p_1 \cdot x_1 + p_2 \cdot x_2 + p_3 \cdot x_3 + p_4 \cdot x_4 \\
	\begin{split}
		\text{troškovi} &= 5 \cdot (c_1 \cdot x_1 + c_2 \cdot x_2 + c_3 \cdot x_3 + c_4 \cdot x_4 \\
		&\quad + c_5 \cdot x_5 + c_6 \cdot x_6 + c_7 \cdot x_7) + c_{51} \cdot x_5 + c_{61} \cdot x_6 + c_{71} \cdot x_7
	\end{split} \\
	\text{zaposleni} &= x_8 + x_9 + x_{10} + x_{11} + x_{12} + x_{13} + x_{14}
\end{align*}
\begin{align*}
	c_1 & = \text{cijena uglja (po kilogramu)} \\
	c_2 & = \text{cijena prirodnog plina (po metru kubnom)} \\
	c_3 & = \text{cijena ogrjevnog drveta (po kilogramu)} \\
	c_4 & = \text{cijena drvnog peleta (po toni)} \\
	c_{51} & = \text{cijena instalacije jedne vjetrenjače} \\
	c_{61} & = \text{cijena instalacije jednog solarnog panela} \\
	c_{71} & = \text{cijena instalacije jedne hidroelektrane} \\
	c_5 & = \text{cijena održavanja jedne vjetrenjače za jednu godinu} \\
	c_6 & = \text{cijena održavanja jednog solarnog panela za jednu godinu} \\
	c_7 & = \text{cijena održavanja jedne hidroelektrane za jednu godinu}
\end{align*}

\subsection*{Ograničenja}
\begin{enumerate}
\item Ograničenja za dostupnost energenata:
\begin{align*}
	x_1 & \leq 225.000.000 \text{ kg} \quad (\text{količina dostupnog uglja}) \\
	x_2 & \leq 254.790.000 \text{ m}^3 \quad (\text{količina dostupnog prirodnog plina}) \\
	x_3 & \leq 245.850 \text{ kg} \quad (\text{količina dostupnog ogrjevnog drveta}) \\
	x_4 & \leq 300.000.000 \text{ kg} \quad (\text{količina dostupnog drvnog peleta})
\end{align*}

	\item Količina proizvedene toplotne energije mora zadovoljiti ukupnu godišnju potrebu od $H = 6.000.000$ kWh:
	\[ A \cdot x_1 + B \cdot x_2 + C \cdot x_3 + D \cdot x_4 + E \cdot x_5 + F \cdot x_6 + G \cdot x_7 \geq H \]
	\item Broj objekata alternativnih izvora koje je moguće instalirati:
	\[x_5 \leq n, \quad x_6 \leq m, \quad x_7 \leq q, \quad n, m, q \in \mathbb{N}\]
	\item Budžetska ograničenja za svaki energetski izvor:
	\begin{align*}
		c_1 \cdot x_1 &\leq \text{budzet\_ugalj} \\
		c_2 \cdot x_2 &\leq \text{budzet\_plin} \\
		c_3 \cdot x_3 &\leq \text{budzet\_drvo} \\
		c_4 \cdot x_4 &\leq \text{budzet\_pelet} \\
		(c_{51}+c_5) \cdot x_5 &\leq \text{budzet\_vjetrenjace} \\
		(c_{61}+c_6) \cdot x_6 &\leq \text{budzet\_panele} \\
		(c_{71}+c_7) \cdot x_7 &\leq \text{budzet\_hidroelektrane}
	\end{align*}
	\item Ograničenja za broj zaposlenih na alternativnim izvorima:
	\begin{align*}
		2 \cdot \frac{x_5}{20} &\leq x_{12} \leq 3 \cdot \frac{x_5}{20} \\
		\frac{x_6}{60} &\leq x_{13} \leq 2 \cdot \frac{x_6}{60} \\
		75 \cdot x_7 &\leq x_{14} \leq 196 \cdot x_7
	\end{align*}
	
	\item Društveni aspekt minimalnog broja zaposlenih:
	\[ x_{8} + x_{9} + x_{10} + x_{11} + x_{12} + x_{13} + x_{14} \geq \text{minimalni\_broj\_zaposlenih} \]
	\item Uslov nenegativnosti varijabli:
	\[ x_1, x_2, x_3, x_4, x_5, x_6, x_7, x_8, x_9, x_{10}, x_{11}, x_{12}, x_{13}, x_{14} \geq 0 \]
\end{enumerate}

\newpage
\section*{SAŽETAK}
\addcontentsline{toc}{section}{Sažetak}

\paragraph{}
Ovaj se projekat fokusira na razvoj linearnog programa s ciljem minimizacije zagađenja zraka u Bosni i Hercegovini, istovremeno optimizirajući troškove i uzimajući u obzir društveni aspekt. Kroz precizno definirane varijable odluke, funkciju cilja i ograničenja, istraživanje obuhvaća širok spektar faktora u energetskom sektoru.

\paragraph{}
Varijable odluke uključuju količine različitih energenata, broj instaliranih alternativnih izvora energije te broj zaposlenih na različitim proizvodnim i instalacijskim pogonima. Funkcija cilja kombinira ciljeve minimizacije zagađenja, troškova i broja zaposlenih, uzimajući u obzir cijene energenata, instalacija i održavanja, te naglašavajući socijalnu odgovornost.

\paragraph{}
Ograničenja uključuju parametre kao što su dostupnost energenata, proizvodnja toplotne energije, ograničenja instalacija alternativnih izvora, budžetska ograničenja te uvjete za broj zaposlenih. Analizom cijena, učinkovitosti resursa i društvenih čimbenika, projekt ima za cilj postići održivo rješenje koje ne samo da minimizira negativne utjecaje na okoliš već i potiče ekonomsku održivost i socijalni razvoj.

\paragraph{}
U konačnici, projekt teži doprinijeti globalnim naporima u smanjenju emisija CO2, pridonoseći očuvanju okoliša i dugoročnoj održivosti energetskog sektora u Bosni i Hercegovini.

\newpage
\section*{IZVORI}
\addcontentsline{toc}{section}{Izvori}

\begin{sloppypar}
\begin{itemize}
    \item \url{www.worlddata.info/europe/bosnia-and-herzegovina/energy-consumption.php}
    \item \url{usitfbih.ba/sumarstvo/}
    \item \url{bhas.gov.ba/data/Publikacije/Saopstenja/2022/ENE_03_2021_Y1_1_HR.pdf}
    \item \url{mycovenant.eumayors.eu/storage/web/mc_covenant/documents/31/gDfNSINefuY8ow-sw1-pA372PsaPZUpE.pdf}
    \item \url{www.quality.unze.ba/zbornici/QUALITY%202005/073-Q05-042.pdf}
    \item \url{vladausk.ba/v4/files/media/pdf/5e3d6631637062.36085396_Plan%20zastite%20kvalitete%20zraka%20USK-a%202017-2022..pdf}
    \item \url{fmeri.gov.ba/media/1564/prilog-1-komponenta-2_metodologija-za-izracun-usteda-energije-smiv.pdf}
    \item \url{www.msb.gov.ba/dokumenti/AB38713.pdf}
    \item \url{bhas.gov.ba/data/Publikacije/Saopstenja/2022/ENE_01_2022_07_1_BS.pdf}
    \item 
    \url{transparentno.ba}
    \item 
    \url{repozitorij.efzg.unizg.hr/islandora/object/efzg%3A6274/datastream/PDF/view}
\end{itemize}
\end{sloppypar}


\end{document}
